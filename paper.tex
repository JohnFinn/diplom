\documentclass[12pt]{article}

\usepackage[utf8]{inputenc}
\usepackage[russian]{babel}
\usepackage{graphicx}
\usepackage{listings}
\usepackage{color}
\usepackage{cite}
\usepackage{titlesec}
\usepackage{svg}
\usepackage{geometry}
\usepackage{indentfirst}
\usepackage{float}

\title{Диплом}
\begin{document}

\maketitle
\pagebreak
\section{Введение}

Существует Компания «Институт Сетевых Технологий».
Она занимается разработками телекоммуникационных продуктов и защищённых систем.

Продуктами компании являются:
\begin{itemize}
    \item средства построения защищённых инфраструктур
    \subitem маршрутизатор доступа гарда 10g
    \subitem криптомаршрутизатор многопротокольный
    \subitem устройство уплотнения каналов «уук-512»
    \subitem межсетевой экран «тродос»
    \subitem система мониторинга и управления сетями и сервисами «нейрон»
    \item Средства телефонной и видеотелефонной связи
    \subitem сервер видеоконференцсвязи
    \subitem ПВТС — пакетная видеотелефонная и телефонная связь
    \subitem интегрированный терминал
    \item Аппаратно-програмные комплексы
    \subitem комплекс корабельной громкоговорящей связи «линия»
    \subitem многофункциональный абонентский терминал (МФАТ) «пеленг»
\end{itemize}

Компания сама разрабатывает Аппаратную архитектуру, разрабатывает програмное обеспечение, производит изделия, и занимается развёртыванием своих продуктах на объектах.

Заказщиками компании являются Министерство обороны РФ, Военно морской флот РФ, а также некоторые частные компании.

Компания успешно конкурирует на Российском рынке с такими компаниями как Cisco и Juniper.

Флагманским продуктом компании является Маршутизатор доступа (МД). В качестве операционной системы он использует не всем привычный linux, а eCos.

Причём eCos не является операционной системой в привычном понимании этого термина
Она похожа на библиотеку по тому, как она используется.
Пользовательский код ссылается на функции из eCos.
При сборке бинарные файлы eCos компануется (линкуется) с пользовательским кодом и загружается с помощью специального загрузчика RedBoot.

В МД реализовано множество сервисов, хорошо знакомым администраторам linux например (Web сервер, SSH сервер), однако DHCP сервер в МД не реализован. Поэтому для назначения IP адресов и других параметров использовались другие компьютеры с DHCP серверами, а на МД был запущен DHCP ретранслятор, выступающий роли посредника, между сервером и клиентами.
% мб побольше про relay написать

Так как протокол DHCP используется повсеместно, в МД тоже понадобилось его реализовать.

% Постановка задачи

% Анализ существующих решений, и Обоснование выбора

% Реализация

\pagebreak
\section{Основная часть}

\subsection{Введение в Архитектуру ПО}

Програмная Архитектура МД - двухсойная.

1й слой отвечает за взаимодействие с железом. 2й за реализацию сетевых протоколов.
Есть три варианта 1го слоя.

\begin{itemize}
    \item для сборки под eCos и загрузки в маршрутизатор.
    \item для сборки под Linux на компьютере разработчика.
    \item для сборки под Windows на компьютере разработчика.
\end{itemize}

\begin{figure}[H]
    \includegraphics[width=\textwidth]{build/router-architecture.png}
    \caption{2й слой архитектуры}
\end{figure}

2й слой предоставляет для разработчика интерфейс, показанный на рисунке.
Есть модули, и объекты управления ими.

Для реализации DHCP сервера, будет добавлен один модуль и один объект управления.
Объект управления будет отвечать за настройки сервера, он будет работать, когда администратор захочет что нибудь изменить в настройках сервера.

Модуль сервера же, в свою очередь будет будет получать пакеты, формировать ответы, и отправлять их в сеть.

\subsection{Описание прикладного процесса}
Компонент будет автоматизировать настройку узлов компьютерных сетей.

Сам процесс выглядит следующим образом:
администратор подходит к компьютеру, открывает программу эмуляции терминала и вручную его настраивает (задаёт IP адрес, прописывает маршруты, возможно что-то ещё). Далее он подходит к следующему компьютеру и тоже его настраивает.

Для автоматизации данного процесса необходимо через web интерфейс или терминал настроить DHCP сервер (описать различные профили настроек и правила их присваивания). Также необходимо настроить присваивание IP адресов по DHCP на клиентах (обычно это по умолчанию и так).

Если DHCP сервер не находится в той же локальной сети, в ней необходимо будет настроить DHCP ретранслятор.

\begin{figure}[H]
    \includegraphics[width=\textwidth / 2]{build/dhcp-simple.png}
    \caption{}
\end{figure}


\subsection{Постановка задачи}

\begin{itemize}
    \item Изучение RFC2131, RFC2132 и других RFC описывающих стандарт DHCP и различных опциональных параметров
    \item Изучение документации на уже существующие DHCP сервера (cisco, isc)
    \item Придумать, как сервер будет конфигурироваться
    \item Проектирование програмной архитектуры и архитектуры данных
    \item Разработка DHCP сервера
    \item Написание документации
\end{itemize}

\subsection{Анализ существующих решений, обоснование выбора.}

Существует множество DHCP серверов с открытым исходным кодом, написанных под linux.
Однако ни один из них не получится использовать, так как в МД в качестве операционной системы используется eCos.

Данная ситуация оставляет только два пути

\begin{enumerate}
    \item Взять один из DHCP серверов с открытым исходным кодом и портировать.
    \item Написать DHCP сервер с нуля.
\end{enumerate}

При портировании придётся потратить много времени на чтение кода.
Также при изменении функциональных требований будет сложнее модифицировать код.

Модель многозадачности в МД не такая как в линуксе.
В линуксе ядро может прервать процесс, в МД процесс должен сам завершится.
Данный факт также усложнит портирование.

В добавок к этому запрещено использовать исключения, а поэтому и STL.

При наисании с нуля будет полное понимание написанного кода, что позволит быстрее реагировать на измениние функциональных требований.

В связи с выше сказанным был сделан выбор в пользу написания DHCP сервера с нуля.

\subsection{Функциональные требования}

Реализация протокола DHCP в соответствии с RFC2131.
\begin{itemize}
    \item Прослушивание сети на наличие DHCP запросов.
    \item Присвоение настроек хоста на основе заданных администратором параметров.
    \item Отправка ответа с присвоенными настройками.
\end{itemize}

Поддержка следующих опций
\begin{description}
    \item[2] Time Offset
    \item[3] Router
    \item[4] Time server
    \item[5] Name server
    \item[6] Domain name server
    \item[7] Log server
    \item[8] Cookie server
    \item[9] LPR server
    \item[10] Impress server
    \item[11] Resource location server
    \item[12] Host name
    \item[13] Boot file size
    \item[14] Merit dump file
    \item[15] Domain name
    \item[16] Swap server
    \item[17] Root path
    \item[18] Extentions path
    \item[19] IP forwarding enable/disable
    \item[20] Non-local source routing enable/disable
    \item[21] Policy filter
    \item[22] Maximum datagram reassembly size
    \item[23] Default IP time to live
    \item[24] Path MTU aging timeout
    \item[25] Path MTU Plateau Table
    \item[26] Interface MTU
    \item[27] All subnets are local
    \item[28] Broadcast Address
    \item[29] Perform mask discovery
    \item[30] Mask supplier
    \item[31] Perform router discovery
    \item[32] Router solicitation address
    \item[33] Static route
    \item[34] Trailer encapsulation
    \item[35] ARP cache timeout
    \item[36] Ethernet encapsulation option
    \item[37] TCP default TTL
    \item[38] TCP keepalive interval
    \item[39] TCP keepalive garbage
    \item[40] Network Information service domain
    \item[41] Network Information servers
    \item[42] Network time protocol servers
    \item[43, 60] Vendor specific Information, Vendor class identifier
    \item[44] NetBIOS over TCP/IP name server
    \item[45] NetBIOS over TCP/IP datagram distribution server
    \item[46] NetBIOS over TCP/IP node type
    \item[47] NetBIOS over TCP/IP scope
    \item[48] X window system font server
    \item[49] X window system display manager
    \item[64] Network Information service+ domain
    \item[65] Network Information service+ servers
    \item[68] Mobile IP home agent
    \item[69] Simple Mail Transport Protocol (SMTP) server
    \item[70] Post Office Protocol (POP3) server
    \item[71] Network news Transport Protocol (NNTP) server
    \item[72] Default world wide web (www) server
    \item[73] Default finger server
    \item[74] Default Internet relay chat (IRC) server
    \item[75] StreetTalk Server
    \item[76] StreetTalk directory assistance (STDA) server
    \item[66] TFTP server name
    \item[67] Bootfile name
    \item[121] Classless route
    \item[118] Subnet selection
    \item[93, 94, 97] PXE options
\end{description}

В протоколе DHCP помимо IP адресов отправлять множество других параметров.
Эти параметры необходимо разбить на несколько логически связанных групп.
Например, TCP параметры. В эту группу должны будут входить 3 параметра.

\begin{itemize}
    \item Значение TTL, используемое при отправке TCP пакетов по умолчанию.
    \item TCP keepalive интервал (сколько ждать перед отправкой keepalive сообщения).
    \item TCP keepalive garbage - отправлять ли специальный октет, для совместимости со старыми реализациями TCP.
\end{itemize}

На такие группы необходимо поделить DHCP параметры.

Администратор, настраивающий DHCP сервер должен иметь возможность создавать экземпляры таких групп, и присваивать им имена, для того чтобы на них затем ссылаться (возможно из нескольких мест одновременно).

\begin{figure}[H]
    \includegraphics[width=\textwidth]{build/first.png}
    \caption{}
\end{figure}

\begin{itemize}
    \item Выбор подсети по адресу ретранслятора
    \item Разделение выдаваемых адресов в подсети по следующему принципу
        \subitem В подсети задается список интервалов IP адресов.
        \subitem У каждого интервала также указывается каждая группа параметров.
        \subitem К каждому интервалу добавляется правила на основе полей пакета.
        \subitem Если правила подошли, адрес и параметры выдаются из этого элемента.
        \subitem Если нет, проверяется следующий элемент.
    \item Помимо описанного выше выделения групп с общими параметрами также нужно, чтобы была возможность присваивать параметры к одному конкретному хосту (например выбирая по mac адресу).
\end{itemize}

\subsection{Архитектура}

\begin{figure}[H]
    \includegraphics[width=\textwidth]{build/second.png}
    \caption{}
\end{figure}

\begin{figure}[H]
    \includegraphics[width=\textwidth]{build/packet-view.png}
    \caption{}
\end{figure}

Тут класс Options\_view написан на шаблонах и гарантирует на этапе компиляции правильность работы с DHCP опциями.

Класс All\_options\_view следит за тем, используются ли поля sname и file в заголовке для хранения опций, и, предоставляя такой же, как и класс Options\_view интерфейс, абстрагирует данную логику.

\begin{figure}[H]
    \includegraphics[width=\textwidth]{build/pinger.png}
    \caption{}
\end{figure}

\pagebreak
\subsection{Тестирование}

Для тестирования были написаны Unit-тесты, с помощью библиотеки googletest.
Помимо них был написан python скрипт для эмуляции DHCP клиентов.

В нём использовался линуксовый механизм сетевых пространств имён.

Были созданы пространства имён, соединены с помощью виртуальных сетевых интерфейсов (veth pair). В каждом из них был запущен либо DHCP клиент от ISC либо DHCP ретранслятор тоже от ISC.

Это дало возможность тестировать на одном компьютере, в близких к реальным условиям.

\includegraphics[width=\textwidth]{build/test-emu.png}

\pagebreak
\section{Заключение}

\end{document}
