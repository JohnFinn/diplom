\documentclass[12pt]{article}

\usepackage[utf8]{inputenc}
\usepackage[russian]{babel}
\usepackage{graphicx}
\usepackage{listings}
\usepackage{color}
\usepackage{cite}
\usepackage{titlesec}
\usepackage{svg}
\usepackage{geometry}
\usepackage{indentfirst}

\title{Диплом}
\begin{document}

\maketitle

\section{Введение}


Одним из продуктов компании "Институт Сетевых Технологий" является МД (маршутизатор доступа), код для него разработан в компании. Этот код компилируется и линкуется с eCos. В МД реализовано множество сервисов, хорошо знакомым администраторам linux например (HTTP сервер, SSH сервер), однако DHCP сервер в МД не реализован (хотя есть DHCP ретранслятор).

Так как протокол DHCP используется повсеместно, в МД тоже понадобилось его реализовать.

\section{Основная часть}

\subsection{Постановка задачи}

\begin{itemize}
    \item изучение RFC2131, RFC2132 и других
    \item Изучение документации на cisco DHCP сервер
    \item Изучение документации на isc DHCP сервер
    \item Проектирование
    \item Разработка
    \item Написание документации
\end{itemize}

\section{Заключение}

\end{document}
