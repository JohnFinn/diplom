\documentclass[12pt]{article}

\usepackage[utf8]{inputenc}
\usepackage[russian]{babel}
\usepackage{graphicx}
\usepackage{listings}
\usepackage{color}
\usepackage{cite}
\usepackage{titlesec}
\usepackage{svg}
\usepackage{geometry}
\usepackage{indentfirst}

\title{Диплом}
\begin{document}

\maketitle
\pagebreak
\section{Введение}


Одним из продуктов компании "Институт Сетевых Технологий" является МД (маршутизатор доступа). В качестве операционной системы он использует не всем привычный linux, а eCos. Причём eCos не является операционной системой в обычном понимании этого термина. Она похожа на библиотеку, так как линкуется с пользовательским кодом и загружается с помощью специального загрузчика RedBoot.

В МД реализовано множество сервисов, хорошо знакомым администраторам linux например (HTTP сервер, SSH сервер), однако DHCP сервер в МД не реализован (хотя есть DHCP ретранслятор).

Так как протокол DHCP используется повсеместно, в МД тоже понадобилось его реализовать.

\pagebreak
\section{Основная часть}

\subsection{Описание прикладного процесса}
Компонент будет автоматизировать настройку узлов компьютерных сетей.

Сам процесс выглядит следующим образом:
администратор подходит к компьютеру, открывает программу эмуляции терминала и вручную его настраивает (задаёт IP адрес, прописывает маршруты, возможно что-то ещё). Далее он подходит к следующему компьютеру и тоже его настраивает.

Для автоматизации данного процесса необходимо через web интерфейс или терминал настроить DHCP сервер (описать различные профили настроек и правила их присваивания). Также необходимо настроить присваивание IP адресов по DHCP на клиентах (обычно это по умолчанию и так).

Если DHCP сервер не находится в той же локальной сети, в ней необходимо будет настроить DHCP ретранслятор.

\subsection{Постановка задачи}

\begin{itemize}
    \item Изучение RFC2131, RFC2132 и других RFC описывающих стандарт DHCP и различных опциональных параметров
    \item Изучение документации на уже существующие DHCP сервера (cisco, isc)
    \item Придумать, как сервер будет конфигурироваться
    \item Проектирование програмной архитектуры и архитектуры данных
    \item Разработка DHCP сервера
    \item Написание документации
\end{itemize}

\subsection{Функциональные требования}

В протоколе DHCP помимо IP адресов отправлять множество других параметров.
Эти параметры необходимо разбить на несколько логически связанных групп.
Например, TCP параметры. В эту группу должны будут входить 3 параметра.

\begin{itemize}
    \item Значение TTL, используемое при отправке TCP пакетов по умолчанию.
    \item TCP keepalive интервал (сколько ждать перед отправкой keepalive сообщения).
    \item TCP keepalive garbage - отправлять ли специальный октет, для совместимости со старыми реализациями TCP.
\end{itemize}

На такие группы необходимо поделить DHCP параметры.

Администратор, настраивающий DHCP сервер должен иметь возможность создавать экземпляры таких групп, и присваивать им имена, для того чтобы на них затем ссылаться (возможно из нескольких мест одновременно).

\includegraphics[width=\textwidth]{build/first.png}

\begin{itemize}
    \item Выбор подсети по адресу ретранслятора
    \item Разделение выдаваемых адресов в подсети по следующему принципу
        \item В подсети задается список интервалов IP адресов.
        \item У каждого интервала также указывается каждая группа параметров.
        \item К каждому интервалу добавляется правила на основе полей пакета.
        \item Если правила подошли, адрес и параметры выдаются из этого элемента.
        \item Если нет, проверяется следующий элемент.
\end{itemize}

\subsection{Архитектура}

\pagebreak
\section{Заключение}

\end{document}
