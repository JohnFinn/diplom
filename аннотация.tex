\documentclass[12pt]{article}

\usepackage[utf8]{inputenc}
\usepackage[russian]{babel}
\usepackage{graphicx}
\usepackage{listings}
\usepackage{color}
\usepackage{cite}
\usepackage{titlesec}
\usepackage{svg}
\usepackage{geometry}

\begin{document}

%  Сбор предварительных аннотаций к ВКР
% Начинается прием предварительных аннотаций к ВКР. Аннотации необходимо предоставить секретарю ГЭК по электронной почте в срок. В предварительной аннотации должна быть отражена следующая информация:

% Предварительная формулировка темы. Можно представить несколько вариантов формулировок, если есть сомнения. При формулировке тем ориентируйтесь на темы прошлого года.
\section{Предварительная формулировка темы}
Разработка компонента, реализующего функционал DHCP сервера для системы управления маршутизатором.

% Компания, в которой осуществляется преддипломная практика и предоставлены материалы для выполнения ВКР.
\section{Компания}
Институт сетевых технологий

% Фамилия, имя, отчество и должность консультанта в компании.
\section{Консультант от компании}
Карапетян Гор Арменович

% Цель ВКР. Ожидаемой целью ВКР является разработка или модификация одного или нескольких компонентов информационных систем, являющихся средствами автоматизации отдельных прикладных процессов или операций.
\section{Цель ВКР}
Существует Аппаратно програмный комплекс маршутизатор доступа. Вся програмная часть реализована в виде одного большого проекта, который после компиляции линкуется с eCos и с помощью RedBoot загружается и исполняется в маршутизаторе.

В проекте реализовано множество модулей (HTTP, SSH сервера). Целью ВКР является добавление модуля в систему, реализующего протокол DHCP.

% Краткое описание прикладного процесса, для автоматизации которого будет использован разрабатываемый компонент, операций, которые будут им автоматизированы и места этого компонента в информационной системе. Необходимо четко указать информационную систему, технологическую платформу и т.п. для которой разрабатывается компонент и способы его взаимодействия с другими компонентами системы.
\section{Описание процесса}
Компонент будет автоматизировать настройку узлов компьютерных сетей.

Сам процесс выглядит следующим образом:
администратор подходит к компьютеру, открывает программу эмуляции терминала и вручную его настраивает (задаёт IP адрес, прописывает маршруты, возможно что-то ещё). Далее он подходит к следующему компьютеру и тоже его настраивает.

Для автоматизации данного процесса необходимо через web интерфейс или терминал настроить DHCP сервер (описать различные профили настроек и правила их присваивания). Также необходимо настроить присваивание IP адресов по DHCP на клиентах (обычно это по умолчанию и так).

Если DHCP сервер не находится в той же локальной сети, в ней необходимо будет настроить DHCP ретранслятор.


Автоматизируемые операции:
\begin{itemize}
    \item Динамическое и статическое присвоение IP адресов
    \item Настройка маршрутов
    \item Присвоение Доменных имён
    \item Множество других настроек, описанных в RFC2132
\end{itemize}

% взаимодействие с другими компонентами
Настраивать компонент можно будет через SSH, или web интерфейс

% В каком объеме предоставлена информация для выполнения ВКР: техническое задание от консультанта, возможность взаимодействия с заказчиком системы, техническая документация на платформу и корпоративные стандарты разработки и т.п.
\section{Информация предоставленная для выполнения ВКР}
Документация по всему проекту.
Рисунок, описывающий желаемые возможности конфигурирования.

% Функциональные требования к разрабатываемому компоненту.
\section{Функциональные требования}
\begin{itemize}
    \item Реализация протокола DHCP
    \item Возможность динамической выдачи ip адресов
    \item Возможность статической выдачи ip адресов на основе mac адреса и поля client-identifier
    \item Возможность создания профилей настроек и правил их присваивания.
\end{itemize}
    
% Описание технологий, каркасов, СУБД, библиотек, готовых компонентов, и др., которые будут использованы при разработке.
\section{Используемые компоненты}
\begin{description}
    \item[Компонент управления маршутизатором] Предоставляет доступ через терминал, а также web интерфейс.
    \item[firewall] один из компонентов маршутизатора, будет использован для фильтрации входящих DHCP пакетов для того чтобы не мешать его тестировать.
    \item[googletest] С++ библиотека для написания юнит тестов
    \item[linux network namespaces] Один из механизмов контейнеризации в linux, позволяющий запускать процессы в отдельных сетевых пространствах имен с отдельными виртуальными сетевыми интерфейсами.
    \item[isc dhcp client] DHCP клиент.
    \item[isc dhcp relay] DHCP ретранслятор.
    \item[ tcpdump] утилита для вылавливания и просмотра сетевых пакетов.
    \item[dhcpdump] утилита для вылавливания и просмотра DHCP пакетов.
\end{description}

\section{Описание методов тестирования и оценки качества разработанного компонента}
Юнит тесты, использующие библиотеку googlestest.

Тестирование в полуавтоматическом режиме с помощью эмуляции различных сетей на компьютере разработчика с помощью сетевых пространств имён (встроенное в linux средство) и DHCP клиента от ISC.

% Описание ожидаемых результатов работы с обязательным указанием формы их представления: приложение или программный компонент, работа которого будет непосредственно продемонстрирована (с реальными данными, с тестовыми данными), модели данных, системной, программной архитектур в виде набора стандартизованных диаграмм (UML, IDEF1x и т.п.) и т.п. 



\end{document}
